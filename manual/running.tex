\chapter{Running \oxdoc}
Although \oxdoc~is a command line utility at its core, the easiest way to work with it is to use
the graphical user interface (GUI).  We do recommend, however, to learn how to use \oxdoc~from the 
command line, because this is more flexible and more convenient if you will
be using \oxdoc~repeatedly. The GUI can be found in the {\tt bin} subdirectory
of your \oxdoc~installation.

\section{Running \oxdoc~using the GUI}



\section{Running \oxdoc~from the command-line}
Using \oxdoc~is rather easy. Generating documentation for an ox project
requires running \oxdoc~and specifying the names of the files you
want to generate documentation from. For example, suppose you have a number
of ox files in a folder. From there, run 
\begin{quote}
\tt oxdoc *.ox
\end{quote}

from the command prompt in that folder. \oxdoc~generates a set
of HTML files, of which {\tt default.html} is the project home file. It also
creates a new style sheet file {\tt oxdoc.css}.

It is advisable to specify an output directory for your project. This can
be done by creating a new \oxdocxml~file in your project directory. See 
Configuration for more information on that.


\subsection{Options}

