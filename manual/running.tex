\chapter{Running \oxdoc}
Although \oxdoc~is a command line utility at its core, the easiest way to work with it is to use
the graphical user interface (GUI).  If you installed \oxdoc~using the setup program in Windows, 
this interface can be accessed from the Start menu. 

\section{Running \oxdoc~using the GUI}

\section{Running \oxdoc~from the command-line}
Using \oxdoc~is rather easy. Generating documentation for an ox project
requires running \oxdoc~and specifying the names of the files you
want to generate documentation from. For example, suppose you have a number
of ox files in a folder. From there, run 
\begin{quote}
\tt oxdoc *.ox
\end{quote}

from the command prompt in that folder. \oxdoc~generates a set
of HTML files, of which {\tt default.html} is the project home file. It also
creates a new style sheet file {\tt oxdoc.css}.

It is advisable to specify an output directory for your project. This can
be done by creating a new \oxdocxml~file in your project directory. See 
Configuration for more information on that.


\subsection{Options}

