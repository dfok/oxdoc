\documentclass{article}

\usepackage{fullpage}

\newcommand\oxdoc{{\tt oxdoc}}
\newcommand\oxdocxml{{\tt oxdoc.xml}}
\newcommand\dvipng{{\tt dvipng}}
\newcommand\bs{{\tt\char`\\}}
\begin{document}


\title{User's Manual for \oxdoc}
\author{Y. Zwols [{\tt yori@1stsign.net}]}
\maketitle

\section{Introduction}

\oxdoc~is a tool for generating API documentation in 
HTML format from comments in ox source code. It is inspired by
Sun Microsystems' Javadoc.







\section{Installation}
Installation is rather easy. Unzip the package {\tt oxdoc.zip} into a
suitable folder, e.g. {\tt c:\bs program files\bs oxdoc}. Next, edit the file
{\tt oxdoc.bat} and alter the path variable in this file. For example,
you may want change the path definition into:

\begin{quote}
\tt set oxdoc=c:\bs program files\bs oxdoc
\end{quote}

Also, it may be helpful to copy the {\tt oxdoc.bat} file into a
folder in your path, e.g. {\tt c:\bs windows}. This way, it is possible to
run \oxdoc~from any folder in the command prompt.

Finally, you may want to edit \oxdocxml~in the {\tt bin}
directory. This file contains general settings for \oxdoc.
See the Configuration section for more information.













\section{Using \oxdoc}

\subsection{Running \oxdoc}
Using \oxdoc~is rather easy. Generating documentation for an ox project
requires running \oxdoc~and specifying the names of the files you
want to generate documentation from. For example, suppose you have a number
of ox files in a folder. From there, run 
\begin{quote}
\tt oxdoc *.ox
\end{quote}

from the command prompt in that folder. \oxdoc~generates a bunch
of HTML files, of which {\tt default.html} is the project home file. It also
creates a new style sheet file {\tt oxdoc.css}.

It is advisable to specify an output directory for your project. This can
be done by creating a new \oxdocxml~file in your project directory. See 
Configuration for more information on that.

\subsection{Writing documentation}
Now you know how to run \oxdoc, it's time to write some comments
in your code. Documentation comments consist of the normal ox comments,
but instead of using {\tt /*} and {\tt */}, we use
{\tt /**} and {\tt **/}. Documentation comments must be placed 
directly above class definitions and function definitions. For example:

\begin{quote}
\begin{verbatim}
/** This file provides the class `Lumberjack`, an abstract representation
of a lumber jack **/

/** The Lumberjack class represents a lumber jack.  A lumber jack can sleep
by using the `Lumberjack::sleep` method.

@example <pre>
decl L = new Lumberjack();
L.sleep(5);
</pre>

@author Y. Zwols **/
class Lumberjack {
   decl wearsWomensClothes;
   isOk();
   sleep(const hours);
};

/** Checks whether this lumber jack is okay
   @returns <tt>TRUE</tt> if this lumber jack is okay, <TT>FALSE</TT> if not.
   @comments In the current implementation, lumber jacks are okay if and only
   if they wear women's clothing.
   @see Lumberjack **/
Lumberjack::isOk() {
   if (this.wearsWomensClothes) 
      return TRUE;
   else
      return FALSE;
}

/** Make the lumber jack sleep for a specified number of hours. A message
is printed to show that he has slept.
   @param hours Number of hours to sleep. Has to be integer.
   @comments The lumber jack cannot sleep more than 24 hours a day. **/
Lumberjack::sleep(const hours) {
   println("The lumber jack is asleep");
   println("(", hours, " hours later...)");
   println("The lumber jack wakes up");
}
\end{verbatim}
\end{quote}

This example shows most of the features. Every documentation block is written
between {\tt /**} and {\tt **/} signs; HTML tags can be used, e.g. to add
markup, or include images. Also, documentation blocks are divided into small
sections by {\tt @} commands. For example, parameters can be described in the {\tt @param}
section, and extra comments are given in the {\tt @comments} section. 

Also, the first sentence of the comment block is taken as a summary of the
documentation block.  This first sentence appears in e.g. the project home
page and the methods table.  \oxdoc~recognizes the first sentence by
scanning for a period followed by a white space.  This may have some undesired
effects when a period in the first sentence doesn't indicate the end of a
sentence, e.g. in the sentence 
\begin{quote}
\tt This class implements Dr. John's algorithm. It solves linear equations.
\end{quote}
Here, the part {\tt This class implements Dr.} will be taken as a summary.
This can be avoided by placing {\tt \&nbsp;} (a non-breaking space)
just after {\tt Dr.}:
\begin{quote}
\tt This class implements Dr.\&nbsp;John's algorithm. It solves linear equations.
\end{quote}

Moreover, it is possible to include any HTML tag. This may be useful for
inclusing of images, or adding more intricate mark up.


\subsection{Types of documentation blocks}
There are different types of documentation blocks.  Depending on the position of the blocks, they are
treated differently.

\begin{itemize}
\item File documentation.  The comment block starting at the first line of a file is reserved as
documentation corresponding to the file. \\
Allowed sections: {\tt @see}, {\tt @example}, {\tt @comments}, {\tt @author}, 
{\tt @version}.

\item Class documentation.  The comment block right before a class definition is treated as
class documentation. \\
Allowed sections: {\tt @see}, {\tt @example}, {\tt @comments}, {\tt @author}, 
{\tt @version}.

\item Method/function documentation.  The comment block right before the body of a function or class method
is treated as documentation for that function or method.
Allowed sections: {\tt @see}, {\tt @example}, {\tt @comments}, {\tt @param}, {\tt @returns}.
\end{itemize}


Although most sections are straightforward text blocks, there are two sections with special features:
\begin{itemize}
\item {\tt @param} describes a parameter or argument of a function. The 
first word after the {\tt @param} keyword is treated as the name of the 
parameter. More than one parameter can be described by adding more {\tt @param} sections.
\item {\tt @see} gives a cross reference. References have to 
match the exact name of other entities. Also, references have to be
separated by commas.
\end{itemize}


\section{Using \LaTeX~formulas in your documentation}
If you have a version of \LaTeX~installed on your computer, it's
extremely easy to add \LaTeX~formulas to your documentation\footnote{If you don't
have \LaTeX, you can download a version for Windows from 
{\tt http://www.miktex.org/}}.
\oxdoc~uses a program called \dvipng~to generate PNG (Portable Network
Graphics) files from \LaTeX~code.  The full installation of MiKTeX comes with a version
of this programs, but other distributions may not have it readily available.  If you
use a non-full installation of MiKTeX, make sure to select \dvipng~during the
installation.

Be sure to read the configuration chapter
to set up \oxdoc~so it can find the programs required for 
this ({\tt latex} and \dvipng).

Once you've set up the connection with \LaTeX, adding formules is done
by putting them between single or double dollar (\$) signs. For example:
\begin{quote}
\begin{verbatim}
<pre>
/** Calculate the OLS estimates for the model $y = X\beta$.
 @returns The OLS estimate $\hat\beta = (X'X)^{-1}X'y$.
 **/
regression(X, y) {
   return invert(X'X)*X'y;
}
</pre>
\end{verbatim}
\end{quote}

Single dollar signs can be used for inline formulae, whereas double dollar signs
can be used for equations on separate lines, analogously to \LaTeX.

\textbf{Note:} \dvipng~generates partially transparent graphics files.  These files are
currently not supported by Internet Explorer 6.0. Supposedly, this will be fixed in version 7.0.
For the time being, I recommended using Mozilla Firefox
to view documentation files that include formulae at this point. At some later point, I will add a
switch to turn of transparent PNG files.

\subsection{Cross-referencing}
Making cross references within comments is done by placing a symbol between ` signs.  It is important to
specify the whole name of the item to be referenced.  Global functions and classes are identified by
their full names (this is case sensitive!) without arguments, and class methods are identified by the
form {\tt classname::method}. For example, if there is a method {\tt isOk()} in the class {\tt Lumberjack},
this method is referenced to by {\tt `Lumberjack::isOk`}.  The same holds for the {\tt @see} sections.
Note that in {\tt @see} sections, no ` signs are to be used.


\subsection{Customizing lay-out}
The lay-out of \oxdoc's output can be controlled by editing the
{\tt oxdoc.css} file in your output directory. Oxdoc creates a default
lay-out file if it is not present, but it won't overwrite changes you make
to that file. 








\section{Configuration}

\subsection{Location of configuration files}
\oxdoc~is configured by means of the file \oxdocxml.
\oxdoc~looks for this file at two locations:
\begin{enumerate}
\item the directory in which {\tt oxdoc.jar}
is located (e.g. {\tt c:\bs program files\bs oxdoc\bs bin});
\item the current working directory.
\end{enumerate}

Whenever available, settings are loaded from these files in that order.
Parameters set in the current working directory configuration file override
the general settings in the \oxdoc~folder.

It is a good idea to put computer-specific settings in the {\tt bin}
directory and project-specific settings in project directories. 

\subsection{Lay-out of \oxdocxml}
A configuration file looks something like this:

\begin{quote}
\begin{verbatim}
<oxdoc>
	<option name="latex"     value="c:\texmf\miktex\bin\latex.exe" />
	<option name="dvipng"    value="c:\texmf\miktex\bin\dvipng.exe" />
	<option name="tempdir"   value="c:\temp\" />
</oxdoc>
\end{verbatim}
\end{quote}

This file specifies values for three options. More option values can be added to
this file as required. See Overview of available settings.

\subsection{Command line configuration}
It is also possible to specify settings through command line arguments
by adding {\tt -parameter=value} to the command line. 
For example,
\begin{quote}
\begin{verbatim}
oxdoc -latex=c:\bin\latex.exe *.ox
\end{verbatim}
\end{quote}
specifies a value for the {\tt latex} setting. The names of the
command line parameters correspond exactly to the settings in 
\oxdocxml.

\subsection{\LaTeX~settings}
\oxdoc~uses \LaTeX~in combination with \dvipng to generate PNG
(Portable Network Graphics) files from formulae within comments. In order to get this
working, you'll need a working distribution of \LaTeX~(e.g. MiKTeX
if you're using Windows) and \dvipng (which comes with MiKTeX). It is then important
to set the paths to the {\tt latex} and \dvipng executables. It is recommended to do this is the \oxdocxml~
file in the {\tt bin} directory of your \oxdoc~installation.

At startup, \oxdoc~checks whether it can find the executables required for \LaTeX~support. If
it can't find one or more of these executables, it automatically turns off \LaTeX~support. In that case,
formulae are literally written in the output. Turning off \LaTeX~support can also be done manually by
setting the {\tt enablelatex} setting to {\tt no}.

It is also possible to specify extra \LaTeX~packages to be included within formulae. This can be done
by specifying the desired packages, separated by commas, in the option {\tt latexpackages}.



\subsection{Overview of available settings}
The following parameters can be set in a configuration file:\medskip

\noindent \begin{tabular}{lp{5in}}
\tt outputdir  & Specifies the directory in which \oxdoc~
writes its output. Defaults to the current working directory. \medskip\\
\tt tempdir & Specifies the directory that \oxdoc~can use for temporary files. Defaults to the current working directory.\\
\tt projectname & Specifies the name of the project. This
name will appear in the project home page.\medskip\\
\tt windowtitle & Specifies the title that will appear
in the window caption in your web browser.\medskip\\
\tt imagepath & Specifies the directory in which \oxdoc~
writes its images. Defaults to the specified output directory. \medskip\\
\tt dvipng & Specifies the full path of the executable 
\dvipng. For MiKTeX users, this can be found under the 
{\tt miktex\bs bin} subdirectory of the MiKTeX installation path. \medskip\\
\tt latex & Specifies the full path of the \LaTeX~compiler. 
For MiKTeX users, this can be found under the 
{\tt miktex\bs bin} subdirectory of the MiKTeX installation path. \medskip\\
\tt enablelatex & Turns on or off \LaTeX~support. Possible values:
{\tt yes}, {\tt no}. Default: {\tt yes} if the required executables can be found. \medskip\\
\tt latexpackages & Specifies what \LaTeX~packages should be loaded
for inline \LaTeX~formulas. These packages are loaded in \LaTeX~files 
through the usual {\tt /usepackage{...}} command. Multiple 
packages can be specified by separating them by commas.
\end{tabular}

\end{document}